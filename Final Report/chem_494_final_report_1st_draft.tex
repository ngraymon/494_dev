\pdfobjcompresslevel=0
\documentclass[12pt,letterpaper, oneside, draft]{article}              % define the basic details of our document
\usepackage{amsmath, graphicx, commath, pgfplots, color, nicefrac}     % get the packages we depend on

\usepackage{fullpage}                                                   % increases the dimensions of the page
\usepackage{setspace}                                                   % this package double spaces our document for editing
\usepackage[backend=biber,style=phys,biblabel=brackets]{biblatex}
\usepackage{braket}
\numberwithin{equation}{section} % numbers equations based on the section they are in

\usepackage[british]{datetime2}

\pgfplotsset{compat=1.11} % this lines forces the package 'pgfplots' to use a specific version so that this document reliably compiles identically across all system

\newcommand{\psiT}{\psi_{\textrm{T}}}
\DeclareMathOperator{\Tr}{Tr}

\newlength{\drop} % for my convenience
\newcommand*{\titleGM}{\begingroup% Gentle Madness
\drop = 0.1\textheight
\vspace*{\baselineskip}
\vfill
	\hbox{%
	\hspace*{0.2\textwidth}%
	\rule{1pt}{\textheight}
	\hspace*{0.05\textwidth}%
	\parbox[b]{0.75\textwidth}{
	\vbox{%
		\vspace{\drop}
		{\noindent\Huge\bfseries Title\\[0.5\baselineskip]
					of the article}\\[2\baselineskip]
		{\Large\itshape Subtitle}\\[4\baselineskip]
		{\Large Neil Raymond}\par
		\vspace{0.5\textheight}
		{\noindent University of Waterloo}\\
		{\noindent \DTMdisplaydate{2015}{07}{27}{0}}\\
		[\baselineskip]
		}% end of vbox
		}% end of parbox
	}% end of hbox
\vfill
\null
\endgroup}

\begin{document}
\pagestyle{empty}
\titleGM


\doublespacing
\section*{Acknowledgement}
\newpage

\section*{Summary}
\newpage

\pagenumbering{roman} % use roman page numbering
\setcounter{page}{1}
\tableofcontents
\newpage

\listoffigures
\newpage

\listoftables
\newpage

\pagenumbering{arabic} % change to arabic page numbering
\setcounter{page}{1}

\addcontentsline{toc}{section}{Introduction}
\section*{Introduction}
A brief outline of what the project is about and why it was started.

\addcontentsline{toc}{section}{istorical}
\section*{Historical}
A brief outline of previous work in the areas before you started.
\newpage

\addcontentsline{toc}{section}{Experimental}
\section*{Experimental}
As much of the experimental details, especially of any new experiments, should be written down, so
someone else could repeat the work or continue it.
\newpage

\addcontentsline{toc}{section}{Results \& Discussion}
\section*{Results \& Discussion}
Equations, tables, illustrations etc..
As long or as shortas necessary to describeand evaluate what was accomplished. 
\newpage

\addcontentsline{toc}{section}{References}
\section*{References}
The last page should list all previous work to which reference was made.
References are best made chronologically throughout the report and listed here in that order.
Authors, publication, and date should be shown in each case.
%\printbibliography
\end{document}