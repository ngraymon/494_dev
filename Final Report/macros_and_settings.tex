\pdfobjcompresslevel=0
\documentclass[12pt,letterpaper, oneside, draft]{article}              % define the basic details of our document
\usepackage{amsmath, graphicx, commath, pgfplots, color, nicefrac}     % get the packages we depend on
\usepackage{fullpage}                                                   % increases the dimensions of the page
\usepackage{setspace}                                                   % this package double spaces our document for editing
\usepackage[backend=biber,style=phys,biblabel=brackets]{biblatex}
\usepackage{braket}
\numberwithin{equation}{section} % numbers equations based on the section they are in

\usepackage[british]{datetime2}

\pgfplotsset{compat=1.11} % this lines forces the package 'pgfplots' to use a specific version so that this document reliably compiles identically across all system

\newcommand{\psiT}{\psi_{\textrm{T}}}
\DeclareMathOperator{\Tr}{Tr}

\newlength{\drop} % for my convenience
\newcommand*{\titleGM}{\begingroup% Gentle Madness
\drop = 0.1\textheight
\vspace*{\baselineskip}
\vfill
	\hbox{%
	\hspace*{0.2\textwidth}%
	\rule{1pt}{\textheight}
	\hspace*{0.05\textwidth}%
	\parbox[b]{0.75\textwidth}{
	\vbox{%
		\vspace{\drop}
		{\noindent\Huge\bfseries Title\\[0.5\baselineskip]
					of the article}\\[2\baselineskip]
		{\Large\itshape Subtitle}\\[4\baselineskip]
		{\Large Neil Raymond}\par
		\vspace{0.5\textheight}
		{\noindent University of Waterloo}\\{\noindent \DTMdisplaydate{2015}{07}{27}{0}}\\[\baselineskip]
		}% end of vbox
		}% end of parbox
	}% end of hbox
\vfill
\null
\endgroup}