\pdfobjcompresslevel=0
\documentclass[12pt,letterpaper,oneside,final,titlepage]{article}               % define the basic details of our document
\usepackage{amsmath, commath, pgfplots, datetime, color, nicefrac}                         % get the packages we depend on
\usepackage{fullpage}                                                   % increases the dimensions of the page
\usepackage{setspace}                                                   % this package double spaces our document for editing
\usepackage[scientific-notation=true]{siunitx}
\usepackage[backend=biber,style=phys,biblabel=brackets]{biblatex}
\usepackage[margin=0.7in]{geometry} 
\usepackage[version=3]{mhchem}    
\usepackage{braket}
\numberwithin{equation}{section} % numbers equations based on the section they are in

\pgfplotsset{compat=1.8} % this lines forces the package 'pgfplots' to use a specific version so that this document reliably compiles identically across all system

\addbibresource{ref.bib}

\newcommand{\psiT}{\psi_{\textrm{T}}}
\newcommand{\eBHf}{e^{\frac{-\beta\hat{H}}{2}}}
\newcommand{\eBH}{e^{-\beta\hat{H}}}
\newcommand{\eTH}{e^{-\tau\hat{H}}}
\newcommand{\emiHt}{e^{\frac{-i\hat{H}t}{\hbar}}}
\newcommand{\eiHt}{e^{\frac{i\hat{H}t}{\hbar}}}

\newcommand{\expb}[1]{\ensuremath{\exp{\left[ #1 \right]}}}
\renewcommand{\vec}[1]{\mathbf{#1}}
\DeclareMathOperator{\Tr}{Tr}

\begin{document}
    \title{Semi-classical calculations of excited state energies}
    \author{Neil Raymond
           \\ \texttt{ngraymon@uwaterloo.ca}}
    \newdate{date_document_was_written}{5}{1}{2015}
    \date{\displaydate{date_document_was_written}}
    \maketitle

\doublespacing
%The format for the report should be a research proposal, where you outline the nature of your research project, the backgrounds to it from the literature, the techniques you will use for your research, and the results you hope to obtain. Maximum of 10 pages total, including a brief summary (200 words), references, and figures. Text should be double spaced, 12 point type. 

\subsection*{Abstract}
A path integral ground state method is used to calculate inital conditions of a semiclassical initial value representation approach for molecular systems. 
It is hypothesised that this approach will yield accurate energy spectra from which excited state energies can be obtained. 

\section{Introduction}

\subsection{Prelude}
Accurate and efficient measurement of molecular properties, such as energy states, is an area of active research in theoretical chemistry. 
The literature shows many different uses for these accurate measurements, and the benefits provided by obtaining them in a timely manner[cite]. 
There are many different methods for calculating these values. 
We propose a combination of a path integral ground state method and a semiclassical method. 
Classical methods, such as molecular dynamics simulations, are regularly used in commercial environments[cite]. 
Such methods are only adequate when the properties of interest do not exhibit significant quantum effects. 
Quantum mechanical analytical methods give exact results but are very slow or intractable for certain problems. 
These are the key factors driving the interest in semiclassical methods; which are based on computationally efficient compuclassical trajectories, but also incorporate quantum effects. 
Semiclassical methods can be both efficient and accurate for certain problems. \newline
It has been shown that ground state energies, for molecular systems, can be calculated using a path integral ground state method with good accuracy and speed[cite]. 
In the research we propose, we will combine a semiclassical initial value representation technique (SC-IVR) with a path integral 
ground state (PIGS) technique to obtain accurate excited state energies~\cite{issack2007semiclassical}~\cite{schmidt2014inclusion}.

\newpage

\subsection{Hamiltonian}
We are interested in the excited state energies of a molecular system.
Clearly we will encounter the Hamiltonian $\hat{H}$ because it is the operator corresponding to the total energy of the system.
We will restrict our investigation to one dimension, although our approach is general and our code will be applicable
to an arbitrary number of degrees of freedom (d.o.f.).
For a system of $N$ particles, where $\hat{V}$ is the potential energy operator that describes the interactions between 
all particles of the system, $m_{i}$ is the mass for particle $i$, and $\hat{p}_{i}$ is the momentum operator, the Hamiltonian is:
\begin{align}
    \hat{H} &= \hat{K} + \hat{V}
    \\      &= \sum_{i=1}^{N}\frac{\hat{p}_{i}^2}{2m_{i}} + \hat{V}
\end{align}


\subsection{Time-dependent Schrodinger}
The Schrodinger equation is a partial differential equation that describes how the quantum state of a physical system changes with time. 
The Hamiltonian generates the evolution of quantum states over time. 
The general time-dependent Schrodinger equation is as follows:
\begin{equation}
    \hat{H}\ket{\psi} = i\hbar\frac{d}{dt}\ket{\psi}
\end{equation}
Knowing the ground state energy of the particles in our system, we obtain the excited state energies via the correlation function.
Using the time dependent Schrodinger equation.
Given a state $\ket{\psi(t)}$ at some initial time (t = 0) and $H$ being independent of time:
\begin{equation}
    \ket{\psi(t)} = \emiHt\ket{\psi(0)}
\end{equation}
We call $\emiHt$ the real-time propagator.


\subsection{Time-independent Schrodinger}
The Schrodinger equation describes the (determinisitic) evolution of the wave function of a particle.
It predicts the probability distributions, but cannot predict the exact result of each measurement.
The time-independent Schrodinger equation describes stationary states. \\
The general form:
\begin{equation}
    \hat{H}\ket{\psi_{n}} = E_{n}\ket{\psi_{n}}
\end{equation}


\subsection{Path integral formulation}
A path integral formulation is a description of quantum theory which generalizes the action principle of classical mechanics.
The classical trajectory is replaced with a functional integral over an infinity of possible trajectories to compute a quantum amplitude.
It is manifestly symmetric between time and space, and allows a scientist to easily change coordinates between very different canonical descriptions of the same quantum system. 
The determinat?


\subsection{Semiclassical initial value representation}
The purpose of the SC approach in the time domain is to find an approximate description of the quantum propagator $\emiHt$ 
in terms of classical trajectories which is valid in the asymptotic limit $\lim_{\hbar \to 0}$.
It does this by propagating clasical trajectories uniquely defined by their initial conditions.
SC-IVR has been shown to provide accurate ZPE, and it can directly be extended to the extraction of excited state energies by simply modifing the initial wave function and increasing the correlation time. 
We use PIGS to modify the initial wave function, without increasing computational time. 
Meaning SC-IVR is capable of describing essentially all quantum effects that are based on the phase space information of the wave function.
The initial value representation (IVR) of the SC method is used because it does not require solving the boundary-value problem (root search), a severe numerical challenge in a multidimensional system. \\
The physical significance of the HK propagator is that for a system in the semiclassical limit, where a typical action is large compared with $\hbar$ the major contribution to the propagator comes from classic trajectories satisfying the correct boundary conditions.




\section{Proposal}
\subsection{Obtaining $\ket{0}$ using PIGS}
We are interested in finding the expectation value of the energy operator in the ground state. 
This is equivalent to finding expectation values of functions with respect to the path integral distributions, which we can do by using PIGS simulations.
We will use a technique know as the Path Integral Langevin Equation (PILE) thermostat to improve the efficiency of our sampling [cite].
Using LePIGS, we sample from a classical distribution which is in some appropriate sense isomorphic to a ground state quantum system approximately represented
using discretized path integrals. 
We will use the software package Molecular Modelling Toolkit (MMTK), which implements the LePIGS method, to run our simulations.
The ground state pseudo-partition function can be expressed in the discretized form as:
\newcommand{\bead}[1]{^{(#1)}}
\begin{align*}
    Z_{\beta,\tau}
    &\propto \idotsint \left[ \prod_{j=0}^{P-1} \dif q\bead{j} \right]
            \cdots \ket{q\bead{M-1}}
            \braket{q\bead{M-1} | e^{-\tau \hat{H}} | q\bead{M}}
            \braket{q\bead{M} | e^{-\tau \hat{H}} | q\bead{M+1}}
            \bra{q\bead{M+1}} \cdots,
\end{align*}
The LePIGS will be preformed in one-dimension, and the dimensional quantities will be used as initial conditions for the SC-IVR method.


\subsection{Using SC-IVR $\&$ HK propagator to obtain dynamics $\emiHt$} 
The properties of interest are the excited state energies of some molecular system. 
Path integral molecular dynamics allows us to simulate a molecular system. 
Given this simulated system how will we obtain the properties of interest?
Correlation functions are versatile functions that can be used to probe dynamical systems.
Correlation functions have the useful property of producing an intensity spectrum after Fourier transformation.
Thus, we can construct a correlation function such that the Fourier transform produces a spectrum, where the peaks
represent the $\delta E_{n}$ between energy levels.
If we know the ground state energy of a molecular system and we know this correlation function, then we should be able to obtain the properties of interest. \\ \\
We can express correlation functions in general as follows:
\begin{equation}
    C(t) = \frac{1}{Z}\Tr[\eBH\hat{B}\eiHt\hat{A}\emiHt]
\end{equation}
Where $\eBH$ is the thermal density operator, $Z = \Tr(\eBH)$ is the canonical partition function 
and $\eiHt\hat{A}\emiHt=\hat{A}(t)$ is the Heisenberg representation of the operator $\hat{A}$.

We are interested in the [not sure of the name?] correlation function so our correlation function is $\braket{\hat{q}(t)\hat{q}}$. 
If we expand:
\begin{align}
    C(t) &= \braket{\hat{q}(t)\hat{q}}
    \\   &= \braket{0|\hat{q}(t)\hat{q}|0}
    \\   &= \braket{0|\eiHt\hat{q}\emiHt\hat{q}|0}
\end{align}
We know that $\hat{H}\ket{\psi_{n}} = E_{n}\ket{\psi_{n}}$ so $\eiHt$ can act on the left and 
then we pull out the constant $e^{\frac{iE_{0}t}{\hbar}}$:\\
%We know that $\hat{H}\ket{\psi_{n}} = E_{n}\ket{\psi_{n}}$ so $\bra{0}\eiHt = \bra{0}e^{\frac{iE_{0}t}{\hbar}}$:
\begin{equation}
    C(t) = e^{\frac{iE_{0}t}{\hbar}}\braket{0|\hat{q}\emiHt\hat{q}|0}
\end{equation}
Now we insert two resolutions of the identity
$\int\int \dif q_{A}\dif q_{B} \ket{q_{A}}\bra{q_{A}}\ket{q_{B}}\bra{q_{B}}$
\begin{align}
    C(t) &= e^{\frac{iE_{0}t}{\hbar}}\int\int \dif q_{A}\dif q_{B} 
    \braket{0|q_{A}}\braket{q_{A}|\hat{q}\emiHt\hat{q}|q_{B}}\braket{q_{B}|0}
    \\   &= e^{\frac{iE_{0}t}{\hbar}}\int\int \dif q_{A}\dif q_{B} 
    q_{A}q_{B}\braket{0|q_{A}}\braket{q_{A}|\emiHt|q_{B}}\braket{q_{B}|0}
    \label{eq:cor_ins}
\end{align}

The real-time propagator $\emiHt$ is troublesome because $\emiHt\ket{q_{B}}$ is intractable. 
To overcome this problem we approximate the real-time propagator.
Two widely used approximations to the real-time propagator are the Herman-Kluck propagator and the Van Vleck-GutzWiller propagator.
We chose to use the Herman-Kluck (HK) propagator instead of the Van Vleck-GutzWiller primarily because:
\begin{enumerate}
    \item The HK propagator is not restricted to small systems based on CT determined by initial and final positions [cite].
    \item The HK propagator has the desired property that the coherent states are localized in coordinate $\&$ momentum space [cite].
\end{enumerate}
The HK propagator:
\begin{equation}\label{eq:HKprop}
    \hat{G}_{HK}(t) = \frac{1}{2\hbar\pi}\int\!\int \!\dif p\,\!\dif q\, 
    \ket{p(t), q(t)}\bra{p,q} R(p(t),q(t))
    \expb{\left(\frac{iS(p(t),q(t))}{\hbar}\right)}
\end{equation}
Substituting the HK propagator into our correlation function~\eqref{eq:cor_ins}:
\begin{equation*}
    C(t) \approx \frac{e^{\frac{iE_{0}t}{\hbar}}}{2\hbar\pi}\int\cdots\int\! \dif p\,\!\dif q\,\!\dif q_{A}\,\!\dif q_{B}\,
    q_{A}q_{B}\braket{0|q_{A}}\braket{q_{A}|p(t), q(t)} \braket{p,q|q_{B}}\braket{q_{B}|0}
    R(p(t),q(t)) \expb{\left(\frac{iS(p(t),q(t))}{\hbar}\right)}
\end{equation*}
For $t=0$
\begin{equation*}
    C(t=0) \approx \frac{1}{2\hbar\pi}\int\cdots\int\! \dif p\,\!\dif q\,\!\dif q_{A}\,\!\dif q_{B}\,
    q_{A}q_{B}\braket{0|q_{A}}\braket{q_{A}|p(t), q(t)} \braket{p,q|q_{B}}\braket{q_{B}|0}
\end{equation*}
\begin{enumerate}
    \item Talk about "fictitious momenta" p and mass $m^{*}$
    \item Talk about two choices $\gamma = \frac{m}{\hbar^{2}\tau}$ and $m^{*} = \frac{m}{P-1}$
    \item Use Euler's equation and the momentum distribution (symmetric around zero) and the fact that sin is odd to make the imaginary part go away
\end{enumerate}


So we have an equation describing the estimator $Q$ and we can simulate a molecular system using PIGS. We have developed code in python that implements the proposed theory. The code uses the MMTK library to run PIGS simulations which are used as the initial conditions for the SC method. In the next section we display some early results of this newly developed code.


\section{Preliminary Results}

\subsection*{Harmonic Oscillator}
The primary test case being investigated is a single electron in a one-dimensional harmonic oscillator. 
The Hamiltonian of our electron, where $m = \num{0.000548579909}$u (units?): 
\begin{align}
    \hat{H} &= \sum_{i=1}^{N}\frac{\hat{p}_{i}^2}{2m_{i}} + \hat{V} \\
    \hat{H} &= \hat{K}(x) + \hat{V}(x) \\
            &= \frac{p^2}{2m} + \frac{1}{2} m \omega^{2} x^{2} 
\end{align}
We will simulate a one-dimensional harmonic oscillator using the mean values of a LePIGS simulation preformed in three dimensions. 
Where $\psiT$ is the approximation to the exact ground state wavefunciton. 
A uniform trial function ($\psiT = 1$) was used due to simplicity of implementation. \\ \\
Here we can see that our initial estimator has some issues:\\
\includegraphics[width=1.0\linewidth]{betaplot.png}
In this graph the black horizontal line represents the analytically determined value.
Each point plotted on the graph represents a distinct simulation at $t=0$ with increasing $\beta$ and constant $\tau$.\\
Our initial estimator converged, but not to the analytically determined value as we had hoped! 
One goal of future work is to find a "scaling factor" such that the estimator converges to the expected value.
Some of our more recent results for the correlation function:\\
\includegraphics[width=1.0\linewidth]{P=1025_1000classicalsteps.pdf}
Here we used a different estimator:
\begin{equation}
    A = 
\end{equation}
A choice was made to set the fictious momentum to zero as an initial condition. This was made because [sampling from entired space?].

\subsection*{Double Well}
The secondary test case will be an electron in a one dimensional double well. 
Some consideration is required in regards to tunneling effects. 
This is because SC-IVR has trouble describing coherent barrier tunneling.

\section{Concluding Remarks}
One Major obstacle is the oscillatory nature of the integral. When sampling, the cancellation between positive and negative terms of the integral results in poor statistics. Therefore more trajectories are required to achieve the same degree of accuracy. To addres this problem we are investigating the use of Filinov filtering.

\newpage

\section{Acknowledgements}
My thanks to Dmitri Iouchtchenko who edited, and contributed significant effort to the mathematical equations presented in this paper. 
My thanks also to Pierre-Nicholas Roy who supervised \& sponsored this research.

\renewcommand*{\bibfont}{\scriptsize}
\printbibliography
\end{document}