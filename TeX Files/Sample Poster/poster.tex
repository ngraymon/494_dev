\pdfobjcompresslevel=0
\documentclass[final,t]{beamer}

\setbeamerfont{itemize}{size=\normalsize}
\setbeamerfont{itemize/enumerate body}{size=\normalsize}
\setbeamerfont{itemize/enumerate subbody}{size=\normalsize}

\usetheme{custom}

\usefonttheme[onlymath]{serif}

\usepackage{amsmath,amsthm,amssymb,latexsym}
\usepackage{exscale}
\usepackage{booktabs, array}
\usepackage[english]{babel}
\usepackage[utf8]{inputenc}
\usepackage[orientation=landscape,size=custom,width=121.9,height=91.44,scale=1.25]{beamerposter}
\usepackage{commath}
\usepackage{mathtools}
\usepackage[version=3]{mhchem}
\usepackage[backend=biber,style=phys,biblabel=brackets]{biblatex}
\usepackage{braket}

\addbibresource{references.bib}

\setbeamertemplate{bibliography item}{\insertbiblabel}

\newcommand{\psiT}{\psi_{\textrm{T}}}

\DeclareGraphicsExtensions{.pdf,.png,.jpg}

\DeclareMathOperator{\Tr}{Tr}
\DeclareMathOperator{\adj}{adj}
\DeclareMathOperator{\atan}{atan}
\DeclareMathOperator{\atant}{atan2}

\newcommand{\expb}[1]{\ensuremath{\exp{\left[ #1 \right]}}}
\renewcommand{\vec}[1]{\mathbf{#1}}

% http://tex.stackexchange.com/questions/433/vertically-center-text-and-image-in-one-line
\newcommand{\vcentered}[1]{\begingroup
\setbox0=\hbox{#1}%
\parbox{\wd0}{\box0}\endgroup}


\title{Semiclassical time correlation functions: \\ Path Integral Ground State (PIGS) \& Initial Value Representation (IVR)}
\author{Neil Raymond, Dmitri Iouchtchenko, and Pierre-Nicholas Roy}
\institute{Department of Chemistry, University of Waterloo, Waterloo, Canada}
\date{November 2014}

\begin{document}

\begin{frame}{}
\begin{columns}[T]


\begin{column}{.31\linewidth}
	\begin{column}{.55\linewidth}
		\vskip-\headheight
		\includegraphics[width=0.5\linewidth]{images/scp_logo}
	\end{column}
	\begin{block}{Motivation}
		\begin{itemize}
			\item Investigating the use of of the Semiclassical Initial Value Representation (SC-IVR) method in formulating correlation functions, for the purposes of finding excited state energies. 
				\vskip1ex
				\begin{itemize}
					\item Excited state energies can be determined from the spectrum produced when applying a Fourier transform to a correlation function
					\item Traditionally using SC-IVR on large complex molecular systems is impractical, but by applying a using time-averaged (TA) SC-IVR this difficulty can be overcome						\cite{issack2007semiclassical}
				\end{itemize}
			\item \textbf{Goals}:
				\begin{itemize}
					\item Use existing method (PIGS in MMTK) to create a pool of initial conditions representative of the true ground state wave function. \cite{schmidt2014inclusion}
					\item Apply the TA SC-IVR formulation of the correlation function to a sample of the pool of initial conditions.
					\item Obtain excited state energies by Fourier transforming the correlation function.
				\end{itemize}
		\end{itemize}
	\end{block}

	\begin{block}{Path integrals}
		\begin{itemize}
			\item Path integral formulation of quantum mechanics allows systems to be represented using chains of classical particles (``beads'')
			\item For the ground state:
				\begin{enumerate}\setlength{\itemindent}{1em}
					\item Propagate a trial function to find the wavefunction: $\ket{0} = e^{-\frac{\beta}{2} \hat{H}} \ket{\psiT}$
					\item Find the density operator: $\hat{\rho} = \ket{0}\!\!\bra{0} = e^{-\frac{\beta}{2} \hat{H}} \ket{\psiT}\!\!\bra{\psiT} e^{-\frac{\beta}{2} \hat{H}}$
					\item Let $\beta = \tau (P-1)$, so
						\begin{align*}
							Z
							&= \Tr{\hat{\rho}}
							= \Braket{\psiT | e^{-\beta \hat{H}} | \psiT}
							= \Braket{\psiT | \left( e^{-\tau \hat{H}} \right)^{(P-1)} | \psiT}
						\end{align*}
					\item Apply Trotter factorization:~\cite{schmidt2014inclusion}
						$
							e^{-\tau \hat{H}}
							\approx
								e^{-\frac{\tau}{2} \textcolor{red}{\hat{V}}}
								e^{-\tau \textcolor{blue}{\hat{K}}}
								e^{-\frac{\tau}{2} \textcolor{red}{\hat{V}}}
						$
					\item Introduce $P$ resolutions of the identity $\int\! \dif q_i \ket{q_i}\!\!\bra{q_i}$ to act as the beads
				\end{enumerate}
			\item We build our path so that the middle bead represented by $M$ has no potential energy acting on it, and the first, last, $M-1$, and $M+1$ beads only have half the potential energy acting on them. \\
			In the following image of two path's $A$ and $B$ potential energy is represented by the dotted red lines. While the "springs" connecting the beads are represented by the wavy blue lines
				\begin{center}
					\includegraphics[width=1.0\linewidth]{images/7-pigs}
				\end{center}
		\end{itemize}
	\end{block}

	\begin{block}{Molecular dynamics}
		\begin{itemize}
			\item \textbf{Path Integral Molecular Dynamics (PIMD)}~\cite{ceperley1995path}
				\begin{itemize}
					\item Fictitious beads are treated as real particles in a real universe
					\item Classical equations of motion are integrated to produce multi-particle trajectory
					\item Trajectory is analyzed using estimators to extract properties
				\end{itemize}
			\item \textbf{Path Integral Ground State (PIGS)}~\cite{sarsa2000path,cuervo2005path}
				\begin{itemize}
					\item Imaginary time propagation is applied to a trial function
					\item Exact result in the infinite propagation time ($\beta \to \infty$) limit
					\item Trial function is arbitrary, but cannot be orthogonal to target ground state
				\end{itemize}
			\item \textbf{Langevin equation Path Integral Ground State (LePIGS)}~\cite{constable2013langevin}
				\begin{itemize}
					\item Introduces Langevin thermostat to PIGS for more efficient sampling
					\item Simulations are NVT (canonical) rather than NVE (microcanonical)
					\item Implemented in the \textbf{Molecular Modelling Toolkit (MMTK)}~\cite{hinsen2000molecular}
					\item Used to find ground state properties of bosonic systems
				\end{itemize}
		\end{itemize}
	\end{block}
\end{column}

\begin{column}{.37\linewidth}
	\begin{block}{Correlation Functions}
		\begin{itemize}
			\item Offer a means of probing dynamical systems
			\item Quantum Mechanical representation: 
				\begin{align*}
					C(t) =\Braket{0 | e^{\frac{i \hat{H} t}{\hbar}} \hat{q} e^{\frac{-i \hat{H} t}{\hbar}} \hat{q} | 0} = \frac{\hbar}{2 m \omega} e^{-i \omega t}
				\end{align*}
				%\begin{align*}
					%C(t) = \frac{1}{Z}\Tr[e^{-\beta \hat{H}} \hat{B} e^{\frac{i \hat{H} t}{\hbar}} \hat{A} e^{\frac{-i \hat{H} t}{\hbar}} ]
				%\end{align*}
				%Where $e^{-\beta \hat{H}}$ is the thermal density operator, $Z = Tr(e^{-\beta \hat{H}})$ is the partition function, and $e^{\frac{i \hat{H} t}{\hbar}} \hat{A} e^{\frac{-i \hat{H} t}{\hbar}} = \hat{A}(t)$ is the Heisenberg representation of operator $\hat{A}$
				\vskip-1ex
			\item $C(t): I(\omega) = \int dt e^{-i \omega t} C(t)$
		\end{itemize}
	\end{block}

	\begin{block}{Semiclassical Initial Value Representation (IVR)}
		\begin{itemize}
			\item "Path" Estimator for $t=0$ case using data collected only form PIGS simulations:
			\begin{align*}
				& A(p, q_{M-1}, q, q_{M+1}) \\
				&= \frac{\sqrt{2}}{P-1} q_{M-1} q_{M+1}
					\expb{
						+ \frac{\beta (P-1)}{2 m} p^2
						+ \frac{\tau}{2}(V(q_{M-1}) + 2V(q_{M}) +V(q_{M+1}))
						- \frac{i}{\hbar}p(q_{M+1} - q_{M-1})
					} \\
				&\qquad\times
					\expb{
						\frac{i}{\hbar} \left( p(t) (q_{M-1} - q(t)) - p (q_{M+1} - q) \right)
					}.
			\end{align*}
			\item Estimator for classical trajectory
			\begin{align*}
				& O(p, q_{M-1}, q, q_{M+1}, t) \\
				&= \frac{\sqrt{2}}{P-1} q_{M-1} q_{M+1}
					\expb{
						-2 \tau E_0
						+ \frac{\beta (P-1)}{2 m} p^2
						+ \frac{i S_{p,q}(t)}{\hbar}
						- \frac{m}{2 \hbar^2 \tau} \left( (q_{M-1} - q(t))^2 - (q_{M+1} - q)^2 \right)
					} \\
				&\qquad\times
					\expb{
						\frac{i}{\hbar} \left( p(t) (q_{M-1} - q(t)) - p (q_{M+1} - q) \right)
					}.
			\end{align*}		
			\item We spawn a classical trajectory from each sampled middle bead $M_{i}$.
			\item We determine  $C(t)$ by averaging the estimator results from the classical trajectory for each $t$. \\
			The following graphic demonstrates how this principle works.
		\end{itemize}
		\includegraphics[width=1.0\linewidth]{images/classical}
	\end{block}
	\begin{block}{Test system}
		\begin{itemize}
			\item Using a single particle inside a harmonic oscillator
				\begin{itemize}
					\item $\hat{H} = \frac{p^2}{2m} + \frac{1}{2} m \omega^{2} x^{2} $
				\end{itemize}
			\item A uniform trial function ($\psiT = 1$) was used due to simplicity of implementation
			\item The  estimator was used to find the energy:
					$$
						\langle E \rangle
						= \frac{\Braket{\psiT | e^{-\beta \hat{H}} \hat{H} | \psiT}}{\Braket{\psiT | e^{-\beta \hat{H}} | \psiT}}
					$$
			
		\end{itemize}
	\end{block}
	\begin{block}{Acknowledgements}
		\vskip-2ex
		\begin{column}{.45\linewidth}
			\begin{itemize}
				\item Chris Herdman
				\item University of Waterloo
				\item NSERC
			\end{itemize}
		\end{column} 
		\begin{column}{.45\linewidth}
			\raggedleft
			\includegraphics[width=0.6\linewidth]{images/nserc}
		\end{column}
	\end{block}				
\end{column}


\begin{column}{.31\linewidth}
	\begin{column}{1\linewidth}
		\vskip-\headheight
		\vskip15ex
		\begin{block}{Results}
			\begin{itemize}
				\item Here we have two plots showing sample results for our path estimator for the $t=0$ case. In each plot data points are drawn from separate PIGS simulations, all run with a $\tau = 0.0625 K^{-1}$ and a varying number of beads, and consequently $\beta$. The horizontal line represents the expected value as derived analytically. 
				\vskip1ex
				\includegraphics[width=1.0\linewidth]{images/21-97h0line}
				\quad
				\item Here in the second plot two sets of data are represented. The first set of data had a low precision and contains simulations with P values in the set $\{21,23 ... 97,99\}$. The second set of data had high precision and contains simulations with P values in the set $\{101,103, ... 197,199\}$
				\vskip1ex
				\includegraphics[width=1.0\linewidth]{images/largelook}
			\end{itemize}
		\end{block}
	\end{column}
	\vskip2ex
	
	\begin{block}{Future work}
		\begin{itemize}
			\item Now that our method works for the $t=0$ case, we need to run the classical trajectories and generate some correlation functions with $t > 0$
			\item Using this SC-IVR approach to calculate the spectrum of two particles in a double well potential Results from this would be a good test of the approaches accuracy and efficiency.
		\end{itemize}
	\end{block}

	\begin{block}{References}
		\renewcommand*{\bibfont}{\scriptsize}
		\printbibliography
	\end{block}
\end{column}

\end{columns}
\end{frame}

\end{document}
